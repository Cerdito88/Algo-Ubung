\documentclass{article}

\usepackage[german]{babel}
\usepackage{graphicx}
\usepackage{amsmath,amsfonts}
\usepackage[utf8]{inputenc}
\usepackage{tikz}
\usepackage{algorithmic}
\usetikzlibrary{automata,arrows}
\usepackage[a4paper, left=2.5cm, right=2.5cm, top=2.5cm, bottom=2.5cm]{geometry}

\begin{document}
\thispagestyle{empty}
\noindent
\begin{minipage}[t]{0.6\textwidth}
\begin{flushleft}
\bf Übungen zur Algorithmen\\
WSI für Informatik\\
Prof. Kaufmann/Güler/Skupin
\end{flushleft}
\end{minipage}
\begin{minipage}[t]{0.4\textwidth}
\begin{flushright}
\bf Sommersemester 2017\\
Universität Tübingen\\
\today
\end{flushright}
\end{minipage}

\setlength{\parindent}{0pt}
\setlength{\parskip}{6pt}

\renewcommand{\labelenumi}{\alph{enumi})}

\vspace*{.3cm}

\begin{center}

\textbf{ \Large  Blatt 2} \\ 
\textbf{Abgabe {\color{red} in Zweier- oder Dreiergruppen} am 04.05.2017}

\end{center}

\vskip 0.3cm

\noindent{\bf Aufgabe 1}  \quad\textbf{(Binäre Suche)} \quad(1+1+1+1{,}5+1{,}5 Punkte)

\noindent
Stellen Sie sich ein Hochhaus mit $n=169$ Stockwerken vor, nummeriert von 1 bis 169. Es geht die Legende um, dass \textbf{ab} einem Stockwerk $1\leq k \leq 169$ die Glühbrine zerbricht und für alle Stockwerke unterhalb unversehrt bleibt. Sie haben nur zwei Glühbirnen zur Verfügung, um herauszufinden, welches $k$ das Stockwerk bezeichnet, ab dem die Glühbirnen bei ihrem Sturz aus dem Fenster zerbrechen. Die {\bf einzige Methode}, die Sie dabei anwenden dürfen, ist, dass Sie eine heile Glühbirne aus beliebigen Stockwerken werfen dürfen, d.h., spätestens dann, wenn die zweite Glühbirne zerbricht, müssen Sie $k$ kennen.
\begin{enumerate}
\item Begründen Sie, warum Sie höchstens $168$ Versuche brauchen.
\item Zeigen Sie, dass Sie höchstens $85$ Versuche brauchen.
\item Zeigen Sie, dass unter $26$ Versuche ausreichen.
\item Wie viele Versuche brauchen Sie minimal? 
\item Verallgemeinern Sie den Ansatz der vorherigen Teilaufgabe für allgemeines $n$ (Anzahl der Stockwerke). 
\end{enumerate}

\noindent{\bf Aufgabe 2}  \quad\textbf{(Implementierung von Suchalgorithmen)} \quad(3+4+4+4 Punkte)\\
Laden Sie die Java-Vorlage und die Datei mit sortierten Zahlen aus dem Moodle herunter und implementieren Sie die fehlenden Funktionen.
\begin{enumerate}
\item Implementieren Sie die Funktion \texttt{readInput(String path)} die einen Dateipfad (\texttt{args[0]}) zur Zahlenliste bekommt und das Feld der Klasse mit den Zahlen füllt.\\
Verwenden Sie dazu die Klasse \texttt{BufferedReader}.
\item Implementieren Sie die Lineare Suche in \texttt{linSearch(int key)}. Der Schlüssel nach dem gesucht wird soll in der \texttt{main(String args)}-Methode über \texttt{args[1]} übergeben werden können.
\item Implementieren Sie die Binäre Suche in \texttt{binSearch(int key)}.
\item Implementieren Sie die Interpolationssuche in \texttt{interpolSearch(int key)}.
\end{enumerate}
{\color{red} \bf Verwenden Sie bei Ihrer Implementierung sinnvolle Variablennamen und kommentieren Sie Ihren Code! Laden Sie Ihre \texttt{Main.java} ins Moodle. Nicht kompilierende Abgaben werden \textbf{mit 0 Punkten} bewertet.}\\
\textit{Schreiben Sie geeignete Testfälle, um Ihre Funktionen zu testen.}


\bigskip
\noindent{\bf Aufgabe 3.}  \quad\textbf{(Rekursionsgleichungen)} \quad(4 + 3 Punkte) \\
Für Algorithmus $A$ ist die Laufzeit in expliziter Form
\[
T_A(n) = (\log_8(n) + 1) \cdot n
\]
gegeben.
Offensichtlich gilt dabei $T_A(1) = 1$.
Die Laufzeit für Algorithmus $B$ ist in Form einer Rekursionsgleichung mit Parameter $\alpha \in \mathbb{R}_+$ gegeben mittels
\[
T_B(n) = 8 T_B\left(\frac{n}{8}\right)  + n^{\alpha}.
\]
\begin{enumerate}
  \item Ermitteln Sie eine geschlossene Form der Rekursionsgleichung für $B$ mittels Induktion und begründen Sie dann damit Ihre Aussage. Nehmen Sie an, dass $n$ eine Achterpotenz und $T_B(1)=1$ ist.
  \item Für welche Werte von $\alpha$ ist der Algorithmus $A$ schneller als $B$? Beweisen Sie Ihre Aussage. 
\end{enumerate}
\end{document}


%%% Local Variables:
%%% mode: latex
%%% TeX-master: t
%%% End:
