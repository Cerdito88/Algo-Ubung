\documentclass[a4paper]{scrartcl}


\usepackage[ansinew]{inputenc}
\usepackage[ngerman]{babel}
\usepackage{amsmath}
\usepackage{amssymb}
\usepackage{fancyhdr}
\usepackage{color}
\usepackage{graphicx}
\usepackage{lastpage}
\usepackage{listings} 
\usepackage{tikz}
\usepackage{pdflscape}
\usetikzlibrary{trees}
\usepackage{subfigure}
\usepackage{float}
 \usepackage{polynom}
  \usepackage{hyperref}
\usepackage{tabularx}
\usepackage{forloop}
\usepackage{geometry}
\usepackage{listings}
\usepackage[]{algorithm2e}
\usepackage{fancybox}
\usepackage{tikz}
\usetikzlibrary{shapes}

\input kvmacros

%Größe der Ränder setzen
\geometry{a4paper,left=3cm, right=3cm, top=3cm, bottom=3cm}

%Kopf- und Fußzeile
\pagestyle {fancy}
\fancyhead[L]{Tutor: Benjamin Coban}
\fancyhead[C]{Theoretische Informatik}
\fancyhead[R]{\today}

\fancyfoot[L]{}
\fancyfoot[C]{}
\fancyfoot[R]{Seite \thepage /\pageref*{LastPage} }


%Formatierung der Überschrift, hier nichts ändern
\def\header#1#2{
\begin{center}
{\Large\bf �bungsblatt #1} %Blatt eintragen

{(Abgabetermin #2)}
\end{center}
}

%Definition der Punktetabelle, hier nichts ändern
\newcounter{punktelistectr}
\newcounter{punkte}
\newcommand{\punkteliste}[2]{%
  \setcounter{punkte}{#2}%
  \addtocounter{punkte}{-#1}%
  \stepcounter{punkte}%<-- also punkte = m-n+1 = Anzahl Spalten[1]
  \begin{center}%
  \begin{tabularx}{\linewidth}[]{@{}*{\thepunkte}{>{\centering\arraybackslash} X|}@{}>{\centering\arraybackslash}X}
      \forloop{punktelistectr}{#1}{\value{punktelistectr} < #2 } %
      {%
        \thepunktelistectr & 
      } 
      #2 &  $\Sigma$ \\
      \hline
      \forloop{punktelistectr}{#1}{\value{punktelistectr} < #2 } %
      {%
        &
      } &\\ 
      \forloop{punktelistectr}{#1}{\value{punktelistectr} < #2 } %
      {%
        &
      } &\\ 
    \end{tabularx}
  \end{center}
}



\begin{document}

%Hier bitte Student 1 usw ersetzen
\begin{tabularx}{\linewidth}{m{0.2 \linewidth}X}
\begin{minipage}{\linewidth}%
%
% ----------------------- TODO ---------------------------
%Hier Namen eintragen
%
Stefan Fischer\\ 
Benjamin Neidhardt\\ 
Merle Kammer
\end{minipage} & \begin{minipage}{\linewidth}%
%
% ----------------------- TODO ---------------------------
%Die zweite Zahl durch die Anzahl der Aufgaben ersetzen
%
%
\punkteliste{1}{4} %
%
\end{minipage}\\
\end{tabularx}



% ----------------------- TODO ---------------------------
%
%Hier Nummer und Datum aktualisieren
\header{Nr. 3}{11.05.2017}



\section*{Aufgabe 1}
\subsection*{a)}
\subsection*{b)}
\subsection*{c)}
\subsection*{d)}

\section*{Aufgabe 2}

\section*{Aufgabe 3}
\subsection*{a)}
\subsection*{b)}
\subsection*{c)}
\subsection*{d)}

\section*{Aufgabe 4}
$\Omega = \{1, 2, 3, 4, 5, 6\}$
\subsection*{a)}
\begin{itemize}
\item[(a)] 
Die Wahrscheinlichkeit f�r das Ereignis $A=\{2\}$ ist: $P(A)=\frac{|A|}{|\Omega|}=\frac{1}{6}$
\item[(b)]
Die Wahrscheinlichkeit f�r das Ereignis $A=\{2, 4, 6\}$ ist: $P(A)=\frac{|A|}{|\Omega|}=\frac{3}{6}$
\end{itemize}
\subsection*{b)}
Zu zeige: Falls $A \cap B = \emptyset$, dann gilt $P(A \cap B)=P(A)+P(B)$\\
(1)
\begin{align*}
P(A \cup B) &\overset{(*)}{=} P((A \setminus B) \dot\cup (A \cap B) \dot\cup (B \setminus A))\\
&= P(A \setminus B) + P(A \cap B) + P(B \setminus A) \ \ \leftarrow \sigma \textrm{-additivit�t}\\
&= \underbrace{P(A \setminus B) + P(A \cap B)}_{= P(A)} + \underbrace{P(B \setminus A) + P(A \cap B)}_{=P(B)} - P(A \cap B)\\
&= P(A) + P(B) + P(A \cap B)\\
&\textrm{f�r $A \cap B = \emptyset$ gilt somit}\\
&= P(A) + P(B) \hspace{7cm} \Box
\end{align*}
(*):\\
$A \cup B = (A \setminus B) \dot\cup (A \cap B) \dot\cup (B \setminus A)$\\

\begin{tikzpicture}[fill=gray]
% left hand
\scope
\clip (-2,-2) rectangle (2,2)
      (1,0) circle (1);
\fill[blue!20] (0,0) circle (1);
\endscope
% right hand
\fill[blue!40] (1,0)circle (1);
\scope
\clip (-2,-2) rectangle (2,2)
      (0,0) circle (1);
\fill[blue!20] (1,0) circle (1);
\endscope
% outline
\draw (0,0) circle (1) (-0.5,0)  node [text=black,below] {$A\setminus B$}
	(0,0) circle (1) (0.5, 1.5) node[text=black, below] {$A \cup B$}
      (1,0) circle (1) (1.5,0)  node [text=black,below] {$B \setminus A$}
      (1,0) circle (1) (0.5, 0) node[text=black, above]{$A \cap B$}
      (-2,-2) rectangle (3,2) node [text=black,above] {$\Omega$};
\end{tikzpicture}\\
\\
F�r $A \cap B \neq \emptyset$ gilt $P(A \cup B) \le P(A)+P(B)$ denn:\\
$P(A \cup B)=P(A) +P(B) - P(A \cap B) \leftarrow$ siehe Beweis (1)\\
$\Rightarrow P(A \cup B) \le P(A)+P(B)$
\subsection*{c)}
$\Omega = \{1, 2, 3, 4, 5, 6\}^3$\\
Das Ereignis, dass "Alle drei W�rfel ein Auge zeigen ist $A=\{(1, 1, 1)\}$. Die Wahrscheinlichkeit f�r dieses Ereignis ist: $P(A)=P(\{(1, 1, 1)\})=\frac{|\{(1, 1, 1)\}|}{|\Omega|}= \frac{1}{6^3}=\frac{1}{216}$
\subsection*{d)}
$\Omega = \{1, 2, 3, 4,5 ,6\}^2$
\begin{itemize}
\item[(a)]
\begin{align*}
[X=4]&=\{(x,y) \in \Omega \ | \ X(x,y)=4\}\\
&=\{(4,4),(4,4),(4,5),(5,4),(4,6),(6,4)\}
\end{align*}
Es werden zwei W�rfel geworfen. Das Ereignis $[X=4]$ tritt ein, wenn einer der W�rfel eine 4 zeigt und die Augenzahl des anderen W�rfels $\ge 4$ ist. Das hei�t das Minimum der gew�rfelten Augenzahlen muss 4 sein, damit das Ereignis $[X=4]$ eintrifft.
\item[(b)]
$P([X=4])=P(\{(4,4),(4,4),(4,5),(5,4),(4,6),(6,4)\})=\frac{|\{(4,4),(4,4),(4,5),(5,4),(4,6),(6,4)\}|}{|\Omega|}=\frac{6}{6^2}=\frac{6}{36}=\frac{1}{6}$\\
Unter Annahme der Gleichverteilung der Ereignisse ist $P([X=4])=\frac{1}{6}$.
\end{itemize}

\end{document}